\begin{center}
    \bf \Large\Heiti{重费米子物理的实验和理论研究进展}
\end{center}
\vskip1.5em
\begin{cnabstract}
		% 在目录中显示摘要
%========================================

重费米子材料是一种强关联电子系统,最近发现多个不等价位点的化合物如\ce{Ce3(Pd/Pt)In11}\\,其拥有两个不等位点Ce1和Ce2,实验表明,超导性来自一个位点,反铁磁性来自另一个位点,针对这种材料,本文构建了多轨道的有效模型,利用行列式量子蒙特卡洛(DQMC)进行模拟研究,本文报道了到目前为止的结果,包括了分析程序的编写。本文将从背景介绍,模型构建,数据程序编写以及初步结果这几个方面来阐述。
\\
\textbf{关键词}: \ \ 重费米子;强关联系统;量子蒙特卡洛计算;反铁磁性;超导态
\end{cnabstract}

%========================================
% 英文摘要

\vskip2em
\begin{center}
    \bf \Large\Heiti{Advances in Experimental and Physical Research of Heavy Fermion Physics}
\end{center}
\vskip1.5em
\begin{enabstract}
    Heavy fermion material is a strongly correlated electronic system. Recently, compounds with multiple unequal sites such as \ce{Ce3(Pd/Pt)In11} have been found, which have two unequal sites Ce1 and Ce2. It is shown that superconductivity comes from one site and antiferromagnetism comes from another site. For this material, an efficient multi-orbital model is constructed for this material, and the simulation study is carried out using determinant quantum Monte Carlo (DQMC). This paper reports the results so far including the writing of the analysis program. This article will describe the background introduction, model construction, data programming and preliminary results.

    
\textbf{keywords}: \ \ Heavy Fermion; Strong connection system%corealtion; 
quantum Metor Carlo; antiferromagnetic; superconductivity;
\end{enabstract}  
