\begin{center}
    \bf \Large\Heiti{重费米子物理的实验和理论研究进展}
\end{center}
\vskip1.5em
\begin{cnabstract}
		% 在目录中显示摘要
%========================================
重费米子材料是一种特殊的过渡金属化合物属于强关联电子系统。电子是一种费米子,
在这种材料中被认为是重电子“重费米子”这个名字来自于费米子的行为,即它的有效质
量好像大于其静止质量。以电子为例,低于特征温度时(通常为10 K),这些金属化合
物中的传导电子表现为它们的有效质量好像是自由粒子质量的1000倍。重费米子材料在
当前的科学研究中发挥着重要的作用,但是作为研究非常规超导、非费米液体行为和量
子临界行为的原型材料。

我们发现\ce{Ce_3PdIn_{11}}在一定温度下,出现了超导相和反铁磁相共存的现象,
这是因为\ce{Ce_3PdIn_{11}}中存在两种不种\ce{Ce}的不等位点。我们将通过量子蒙特
卡洛方法对此进行模拟。
\\
\textbf{关键词}: \ \ 重费米子;强关联系统;量子蒙特卡罗计算;反铁磁性;超导态;共存
\end{cnabstract}

%========================================
% 英文摘要

\vskip2em
\begin{center}
    \bf \Large\Heiti{Advances in Experimental and Physical Research of Heavy Fermion Physics}
\end{center}
\vskip1.5em
\begin{enabstract}
    Heavy fermion materials are a special kind of transition metal compounds belonging to strongly correlated electron systems. Electrons are a kind of fermions,
The "heavy fermions" that are considered heavy electrons in this material get the name from the fermion's behavior that its effective mass appears to be greater than its rest mass. In the case of electrons, below a characteristic temperature (usually 10 K), conduction electrons in these metal compounds behave as if their effective mass is 100 times the mass of the free particle. Heavy fermion materials in
It plays an important role in current scientific research, but as a prototype material for studying unconventional superconductivity, non-Fermi liquid behavior and quantum critical behavior.

We found that \ce{Ce_3PdIn_{11}} coexisted superconducting phase and antiferromagnetic phase at a certain temperature,
This is because there are two kinds of unequal sites of \ce{Ce} in \ce{Ce_3PdIn_{11}}. We will simulate this by a quantum Monte Carlo method.

    
\textbf{keywords}: \ \ Heavy Fermion particle; Strong connection system; 
quantum Mentor Carlo; antiferromagnetic; superconductive;
\end{enabstract}  
